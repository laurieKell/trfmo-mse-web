
Recommendation [10-04] states “In 2012, and thereafter every three years, the SCRS will conduct a stock assessment for bluefin tuna for the western Atlantic and eastern Atlantic and Mediterranean and provide advice to the Commission on the appropriate management measures, inter alia, on total allowable catch levels for those stocks for future years.” 

The Atlantic-wide Research Program for Bluefin tuna (GBYP) and various National programs have produced, and continue to produce, a great deal of new information on the biology and fisheries for bluefin tuna. In preparation for the planned 2015 assessment, time and resources of the SCRS are thus required to validate these data and to incorporate them in the ICCAT database as well as working on updated biological parameters and new modeling approaches. Therefore, the SCRS planned for several meetings in the 2012 work plan. The first two took place in 2013 and aimed at updating the biological parameters and comparing various modeling platforms. For 2014, the SCRS plans a data preparatory meeting to incorporate the new catch and effort information in ICCAT databases and continuing working on new modeling platforms. 

Recommendation [12-03] for the eastern Atlantic and Mediterranean bluefin tuna states “In 2014 the SCRS will conduct an update of the stock assessment and provide advice to the Commission.../... Furthermore, the SCRS shall work towards the development of new assessment modeling approaches and inputs, in a view to minimize uncertainties, which shall be used in a stock assessment in 2015 and thereafter every three years.” 

The Group expressed concern regarding the above Recommendation, mostly because the SCRS may have not the resources to update the assessment of the East Atlantic and Mediterranean bluefin tuna in 2014 while also undertaking the difficult task of preparing for the 2015 assessment. In this regard, the Commission may wish to consider how the limited resources of the SCRS can be most effectively utilized. This dilemma has been debated by the SCRS, which considers that any update of East Atlantic and Mediterranean stock assessment should include updated Task I and II databases. To accommodate priorities to improve the scientific advice by 2015 and last commission request, the SCRS proposes the following work plan for 2014:

\begin{enumerate}[1.]

\item Update fishery indicators in accordance Rec. [12-03], paragraph. 50 (to be done during the annual species group meeting preceding the SCRS plenary in Madrid in 2014).
\item Conduct an Inter-sessional Preparatory Workshop in early 2014 (6 days) that will focus on the following: 
\begin{enumerate}[a)] 
 \item Revise Task II by validating and integrating the catch at size statistics with new information from farms and other sources of information.
 \item Revise Task I (aggregated catch, by gear/fleet) data by including new sources of information from BCDs and trade statistics (i.e. outputs from experts contracted by the GBYP).
 \item Review tagging past and recent data for bluefin tuna.
 \item Complete outstanding tasks from the Biological Parameters meeting in Tenerife (age-length relationships, morphometric conversions, natural mortality, reproduction, etc.).
\end{enumerate}
\item Continue a series of workshops and related activities (to be sponsored by the GBYP and various national programs) in accordance with recommendations from the Biological Parameters Meeting (Tenerife) and the Bluefin Methods meeting (Gloucester) including:
\begin{enumerate}[a)] 
\item Establish a reference collection for otoliths and hard parts and calibrate age estimates among readers.
\item Larval biology workshop.
\item Continue the development of new modeling platforms that can better take into account various sources of uncertainties.
\end{enumerate}
\end{enumerate}

There is thus a considerable amount of work to be done in 2014, i.e., validating and incorporating 10,000s of new files into the current ICCAT databases, calibrating and updating all the size and age conversion methods and continuing the development of new modeling platforms.

Therefore, if the 2014 Bluefin Tuna Species Group is able to incorporate these new sources of information into Task I and II databases and to complete the biological parameters by June 2014 the), the SCRS proposes that an additional inter-sessional meeting be planned in September 2014 to update the eastern Atlantic bluefin tuna stock assessment.  However, even if the new data are available this stock assessment is unlikely to reduce substantially most of the unquantified uncertainties. 

However, if the 2014 bluefin tuna species Group cannot complete these tasks by the end of the workshop (or slightly later), the SCRS proposes postponing the East Atlantic bluefin tuna stock assessment to 2015, as previously planned. 

Nonetheless, if the Commission still considers updating the 2014 assessment to be of higher priority, then most of the activities under item (2) and some of item (3) above should be postponed to 2015 and the corresponding 2015 assessment would be postponed until 2016. Note that the eastern Atlantic bluefin tuna stock assessment is postponed to 2016; this will have some implication on the western Atlantic bluefin tuna stock assessment due to mixing issues. 


\end{document}

\subsection{Peer Review Paper 1}

\textit{Paper written as part of GBYP Phase III, intention is to sumbit to a peer review journal this year\\}

Addresses tasks  3.\\

\textbf{Risk Assessment Eliciting uncertainties in GBYP}

Leach, A.W., Levontin, P., Holt, J. and Mumford, J.D.


The GBYP Bluefin tuna rebuilding plan uses stochastic projections that do not capture all the
uncertainty associated with stock assessment/ management variables. This could mean that the
outcomes predicted by the projections are more optimistic than those achieved in reality, or at
least may be different from reality. A methodology was sought to capture stakeholder
perceptions of particular uncertainties that should be more effectively included in stock
assessments of Bluefin tuna and then to provide preliminary quantification of their relative
importance in terms of their impact on achieving management objectives. Ultimately, this will
allow risk-based scenarios to be specified for the operating mode of a Management Strategy
Evaluation approach, and enable SCRS and the GBYP Steering Committee to prioritize
research. A spreadsheet-based questionnaire was developed and tailored further to capture
measures of stakeholder uncertainty for risk-related variables, assumptions and hypotheses
(identified from the literature and stakeholder consultation in Madrid, during 2011 and 2012).
In this phase we approximately doubled the number of scientists in the stakeholder group
consisting of ‘experts’, expanded the NGO stakeholder group, and have distributed
questionnaires among the group representing managers and decision makers. The latest group
has been the slowest to respond, in particular, we have not heard back yet from the
representatives of the EC. Therefore, we can only offer a limited analysis at this point, but one
which already highlights significant areas of both consensus and lack of consensus.


\subsection{Peer Review Paper 2} 

\textit{Paper written as part of GBYP Phase III, intention is to sumbit to a peer review journal this year\\}

Addresses tasks  3.\\

\textbf{The spectre of uncertainty in management of exploited fish stocks: the illustrative case of Atlantic Bluefin tuna}

Jean-Marc Fromentin1, Sylvain Bonhommeau, Haritz Arrizabalaga, Laurence T. Kell. 


The recent overexploitation of East Atlantic and Mediterranean bluefin tuna stock has been well documented in the medias where it has become the archetype of overfishing and general mis-management. Beyond the public debate, the crisis also highlighted how the interactions between science and management can change through time according to the awareness of the public opinion. To reflect these issues, we describe the history of Atlantic bluefin tuna overfishing and the recent improvement following the implementation of the rebuilding plan. We then summarise the major uncertainties that undermine the current scientific advice and stress the importance of reducing their impacts by improving knowledge and developing a robust management and scientific advice framework. We also discuss how uncertainty was used by different lobbies to discredit science-based management and to push their own agendas. The recent improvement in Atlantic Bluefin tuna stock status also shows that despite uncertainty the management of a heavily exploited fish stock can still be successful when there is a real and strong political will. We conclude by advocating for the implementation of a scientific quota that should be part of the management framework to support the scientific advice.


\subsection{GBYP Contract I}

\textit{Deliverable under Phase IV\\}

Addresses tasks  4 \& 5\\

May be either an SCRS or peer review paper.

\textbf{Specifying and weighting scenarios for MSE robustness trials}

Leach, A.W.1, Levontin, P.1, Holt, J.1 and Mumford, J.D.1

The GBYP Bluefin tuna rebuilding plan uses stochastic projections that do not capture all the uncertainty associated with stock assessment/ management variables.  This could mean that the outcomes predicted by the projections are more optimistic than those achieved in reality.  A methodology was sought to capture stakeholder perceptions of particular uncertainties that should be included in stock assessments of Bluefin tuna and then to provide preliminary quantification of their relative importance in terms of their impact on achieving management objectives. Ultimately, this will allow risk-based scenarios to be specified for operating models of a Management Strategy Evaluation approach, and enable SCRS and the GBYP Steering Committee to prioritize research.  Given that the combinations of scenarios for inclusion in an MSE grow exponentially with each extra variable, it will not be possible to evaluate the quantitative impact of all sources of uncertainties identified, or even prioritised. Discussions with modellers were needed to reduce the initial list to those variables most amenable for further evaluation using a simpler modelling approach such as elasticity analysis. In elasticity analysis the proportional change of the key operating model (OM) outputs, summarised in an objective function, is calculated relative to changes in the input variable or a base-case scenario. Having determined which of the uncertainties have greater impact on the objective function in the elasticity analysis, discussions can be initiated with the stakeholders to elicit which interactions among the shortlisted uncertainties should have priority for further quantitative investigations. Finally, a representative ‘reference’ set of operating models can be selected based on analysis of interactions among uncertainties. The plausibility weights for this reference set of OMs provide another opportunity to engage stakeholders, and to elicit their views as to how robustness trials with the MSE should be ‘tuned’. Having thus established an MSE framework, other sources of uncertainty from the qualitative analysis stage can be quantitatively addressed but it is still unlikely that every single one can be given a quantitative treatment.  Therefore, elicitation process will also serve to document what is missing from the quantitative risk assessment, giving a more transparent and comprehensive view of uncertainties in the scientific advice to managers and other stakeholders.

\subsection{Training}

Addresses tasks  7.a\\

http://www.ices.dk/news-and-events/Training/Pages/ICES-ICCAT-Management-Strategy-Evaluation-%28including-FLR%29.aspx


\textbf{Population Hypotheses and Stock Assumptions for North Atlantic Bluefin Tuna.}\\
Haritz Arrizabalaga,  Jean-Marc Fromentin, Laurence T. Kell, Lisa A. Kerr, Ai Kimoto, Clay Porch, David H. Secor.\\
                     
\gls{BFT} is currently managed as two separate eastern and western stocks. However, tagging, 
micro-chemistry and genetic studies suggest that the stocks mix and within a stock there are
sub components. Differences are also assumed in life history parameters, e.g. maturity 
and natural mortality, used in their stock assessments.

We review population hypotheses and discuss the importance of the population hypotheses 
for the current and alternative management framework. We then specify  
equations to map the population components into the current and alternative 
management units and discuss how to construct an \gls{OM} to be used 
as part of a \gls{MSE}.


\section{Tasks Based on MSE Workplan}

\subsection{2013}

\begin{enumerate}
\item Discussion of alternative mixing structures in broad terms SCRS paper with key contributors. 
%\textbf{[SCRS paper 1]}
\item Clarification of standard inputs to standard separate west/east assessments. Use ICCAT meeting to facilitate with those most familiar with data (Terms of Reference document). Table of information available 
%\textbf{[?]}
\item Clarification on data availability for mixing and stock structure related data for more complex stock assessments. Genetic, microconstituents, tags (archival, conventional, other) 
%\textbf{[SCRS paper 1]}
\item Identification of major sensitivities for both separate and mixed stock assessments (e.g., M, fecundity schedule, SRR and alternative mechanism of population regulation) 
%\textbf{[SCRS paper 1, Peer Review Paper 1, Peer Review Paper 2]}
\item Use Risk assessment paper on qualitative identification of uncertainty (written under GBYP modelling contract) to inform OM scenarios, i.e., SCRS paper with key contributors. 
%\textbf{[Peer Review Paper 1, GBYP Contract I]}
\item Identification of those who will be taking both assessment approaches further forward Consistent, core group over a multi-year timeline 
%\textbf{[?]}
\item Support capacity development for conduct, understanding and use of MSE in adoption of Harvest Control Rules for the Atlantic bluefin fisheries through:
\begin{enumerate}
\item ICES/ICCAT MSE training in (Dec. 2013) to facilitate capacity building for CPC scientific delegations; 
%\textbf{[ICES/ICCAT Training Course]}
\item Take advantage of GEF/FAO Areas Beyond National Jurisdiction Tuna Program funds intended to accelerate the joint tuna RFMO working group for MSE development and management /stakeholder / science (jargon-free) dialogue. 
%\textbf{[wait and see]}
\item Conduct a ‘side event’(SCRS Chairman to co-ordinate) at the 2013 Commission meeting open to CPCs and stakeholder groups, drawing upon the experience at CCSBT to initiate the management / science/ stakeholder dialogue. 
%\textbf{[!]}
\end{enumerate}
\end{enumerate}

\subsection{2014}


\begin{enumerate}
\item For bluefin session
\begin{enumerate}
\item Eastern assessment update 
\end{enumerate}

\begin{enumerate}
\item Post-bluefin; ideally follows bluefin session
\item Review of updated separate assessment approaches
\item Review of initial mixed stock models and refinement of alternative mixing structure scenarios
\item Tool for visualizing movement
\item Meeting including stakeholders (finalise at 2013 Commission meeting)
\end{enumerate}
\end{enumerate}

\newpage
\section{Deliverables}

\subsection{Alternative Mixing and Stock Structures} 

\textbf{Action so far}
An outline draft has been agreed with co-authors which will addresses tasks 1 and 3,
but no work has been done so far-

\subsection{Assessent Inputs} 


\textbf{Action so far}
Nothing has been done, although this has important consequences for the update in 2014 and the MSE.

\subsection{Data on Alternative Mixing/Stock Structures} 

See paper outline above

\subsection{Major Sensitivities} 

Two peer review papers have been written (one accepted and the other submitted),
two SCRS papers prepared for the data preparatory.

\subsection{Risk assessment} 
Report produced under the GBYP, which may be turned in to a paper for use in
formulating scenarios for use in the MSE.

\subsection{Responsible Persons for Taking Assessment Approaches Forward} 

Work was conducted under the GBYP using the iSCAM statistical catch-at-age model.

\subsection{Support} 

\subsubsection{ICCAT/ICES Training} 
Cancelled.

\subsubsection{GEF} 
The GEF project has now started.

\subsubsection{Commission Side Event} 
Ask Josu Santiago for feedback.
