\documentclass[a4paper,10pt]{article}
\usepackage[utf8]{inputenc}
\usepackage{hyperref}
\usepackage[authoryear, round]{natbib}

% Title Page
\title{Glossary of Terms}
\author{}

\begin{document}
\maketitle

There are various glossaries related to MSE, stock assessment and risk.
 
\section{MSE}
  The main reference for terminology is in \href{http://icesjms.oxfordjournals.org/content/64/4/618.abstract}{\cite{rademeyer2007tips}}
  
\section{tRFMOs}

Some of the tRFMOs have their own glossaries
  %CCSBT \url{http://cran.r-project.org/web/packages/kobe}\\ 
  %IATTC \url{http://cran.r-project.org/web/packages/kobe}\\
  ICCAT \url{http://www.iccat.int/Documents/SCRS/Other/glossary.pdf}\\
  IOTC  \url{http://www.iotc.org/files/proceedings/2012/sc/IOTC-2012-SC15-INF03.pdf}\\
  %WCPFC \url{http://cran.r-project.org/web/packages/kobe}\\
  
\section{Risk}

  The International Organization for Standardization (ISO) is the world’s largest developer of voluntary International Standards. 
  International Standards give state of the art specifications for products, services and good practice, helping to 
  make industry more efficient and effective. Developed through global consensus, they help to break down barriers to international trade.
  They have a standard for Risk Management see

  ISO 31000:2009 \url{http://www.iso.org/iso/catalogue_detail?csnumber=43170}


\bibliographystyle{abbrvnat}
\bibliography{refs}

\end{document}          
